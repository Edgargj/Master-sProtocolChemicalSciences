\documentclass[11pt]{article}
\usepackage{graphicx}
\usepackage[spanish]{babel}
\usepackage{multirow}
\usepackage{cite}
\usepackage{textpos}
\usepackage{hyperref}
\usepackage{comment}


\begin{document}
\begin{titlepage}

\begin{textblock}{3}(0,11.5)
   \begin{tabular}{ccc}
      Puebla, Pue. & \phantom{MMMMMMMMMMMMMMMMM} & Mayo de 2024\phantom{MMMMMM}
   \end{tabular}
\end{textblock}

\centering
\phantom{M}

\vspace{-20mm}
{\includegraphics[width=4cm]{buap} \par}
\vspace{5mm}
{\bfseries\huge Benem\'{e}rita Universidad Aut\'{o}noma de Puebla \par}
\vspace{10mm}
{\Large Facultad de Ciencias Qu\'{i}micas\par}
{\Large Centro de Qu\'{i}mica-Instituto de Ciencias \par}
\vspace{5mm}
{\Large Posgrado en Ciencias Qu\'{i}micas\par}
\vspace{15mm}
{\textbf{``La dimensi\'{o}n fractal de masa como medida caracterizadora\\de la estructura de las prote\'{i}nas''} \par}
\vspace{15mm}
{\large Protocolo de Tesis de Maestr\'{i}a\par}
\vspace{5mm}
{\Large Lic. en Qu\'{i}mica \par}
{\Large Edgar Garc\'{i}a Ju\'{a}rez \par}
\vspace{20mm}


\begin{tabular}{p{0.45\textwidth}cp{0.45\textwidth}}
  \cline{1-1} \cline{3-3} \\
  \centering Director de tesis \\ Dr. J. Manuel Solano Altamirano & & \centering Co-Directora de tesis\\ Dra. Viridiana Vargas Castro 
\end{tabular} 

\end{titlepage}

%-----------------------------------------------------------------------------%
% Cover Page
%-----------------------------------------------------------------------------% 
\clearpage

\section{Introducci\'{o}n}

Los cristal\'{o}grafos han observado durante mucho tiempo que las prote\'{i}nas son pol\'{i}meros muy compactos. Sin embargo, la estructura nativa que se captura en un cristal es, aunque representativa, solo es una de las muchas que una prote\'{i}na puede adoptar durante el curso de su funci\'{o}n en la c\'{e}lula viva. La noci\'{o}n de que las prote\'{i}nas son simplemente objetos tridimensionales extremadamente compactos puede ser demasiado simple. De hecho, se ha señalado desde hace tiempo la posibilidad de que las prote\'{i}nas puedan estar mejor caracterizadas por la geometr\'{i}a fractal en lugar de objetos tridimensionales compactos \cite{Enright2005}.
En adici\'{o}n, el problema de cuantificar las diferencia entre dos estructuras de una misma prote\'{i}na no es trivial y contin\'{u}a evolucionando. Debido a que el aspecto y distribuci\'{o}n de las prote\'{i}nas es aparentemente desordenado, la dimensi\'{o}n fractal puede sintetizar toda la informaci\'{o}n que contienen estos sistemas. Desde hace varios años la dimensi\'{o}n fractal de masa se ha utilizado para identificar la distribuci\'{o}n de distintos patrones, \textit{v. gr.}, sistemas magnetoreol\'{o}gicos (piedras o polvos magnetizados o geles envejecidos) \cite{Carrillo2003}. Lo anterior sucede porque cuando se mide la dimensi\'{o}n fractal de masa a diferentes escalas, se puede obtener  informaci\'{o}n acerca de los patrones de agregaci\'{o}n; adem\'{a}s, se ha observado la aparici\'{o}n de c\'{u}mulos o clusters con propiedades multifractales, estos c\'{u}mulos tambi\'{e}n produjeron varias generaciones de clusters secuencialmente, mostrando un comportamiento lineal.

Anteriormente, autores como Matthew \textit{et al.} \cite{Enright2005} han calculado la dimensi\'{o}n fractal de varias prote\'{i}nas de forma global, sin el uso de varias etapas de agregaci\'{o}n. Por lo tanto, es necesario analizar c\'{o}mo evoluciona la dimensi\'{o}n fractal de masa en el tiempo. El an\'{a}lisis multifractal encontrado por Carrillo \textit{et al.} \cite{Carrillo2003}, aplicado a diferentes tipos de estructuras (no simplemente a geles envejecidos), podr\'{i}a servir para extraer informaci\'{o}n de los patrones de agregaci\'{o}n de otros objetos en diferentes etapas. Se sabe que, en las prote\'{i}nas existe m\'{a}s de una etapa de agregaci\'{o}n o de conformaci\'{o}n, esto plantea la siguiente pregunta; ¿Se podr\'{i}a detectar estos patrones de agregaci\'{o}n en las prote\'{i}nas midiendo la dimensi\'{o}n fractal de masa en este esquema de objetos multifractales? De ser cierto lo anterior, ¿ser\'{i}a esta medida lo suficientemente precisa para distinguir entre dos prote\'{i}nas con secuencias gen\'{e}ticas similares pero estructuralmente diferentes? Para responder a estas interrogantes, es fundamental desarrollar una herramienta computacional que nos permita calcular la dimesi\'{o}n fractal de masas y llevar a cabo la pruebas exhaustivas.

Es importante resaltar que, de ser cierto lo anterior, podr\'{i}a abrir la puerta a futuras aplicaciones en enfermedades. No obstante, a\'{u}n no se sabe con certeza si es posible observar la multifractalidad en prote\'{i}nas, y esta incertidumbre es parte de lo que se busca analizar.


\section{Antecedentes}

\subsection{Fractal y dimesi\'{o}n fractal}
\label{subsec:subseccion2.1}


Existen objetos que tienen la propiedad de que su geometr\'{i}a tenga una apariencia fragmentada o con una morfolog\'{i}a aparentemente irregular y que no cambia a cualquiera que sea la escala a la que se analice; a estos objetos se les conoce como fractales. Los fractales est\'{a}n presentes en un gran n\'{u}mero de sistemas complejos que van desde agregados microsc\'{o}picos hasta c\'{u}mulos de galaxias, por lo tanto, existen objetos que se pueden describir en t\'{e}rminos de su geometr\'{i}a fractal \cite{Vicsek1992}. 

La autosimilitud o invariancia de escala es la propiedad m\'{a}s importante en la geometr\'{i}a fractal porque puede modelar la complejidad de cualquier objeto. La dimensi\'{o}n fractal \textbf{(D)} es el \'{i}ndice num\'{e}rico que cuantifica esta complejidad, midiendo la tasa de adici\'{o}n de detalles estructurales con una mayor ampliaci\'{o}n.

En la literatura cient\'{i}fica puede encontrase una gran variedad de m\'{e}todos para determinar la dimensi\'{o}n fractal, algunos de ellos son la dimensi\'{o}n fractal Hausdorff-Besicovitch, la dimensi\'{o}n fractal Minkowski y la dimesi\'{o}n de fractal por conteo de cajas. La dimensi\'{o}n fractal puede representarse como:

\begin{equation}
 D = \frac{\ln(N)}{\ln(1/\varepsilon)}
\end{equation}

Donde \(N\) es el n\'{u}mero de unidades constitutivas en la generaci\'{o}n actual que han emergido de una unidad constitutiva en la generaci\'{o}n anterior, cuando se avanza en la construcci\'{o}n del patr\'{o}n una etapa m\'{a}s adelante, o el n\'{u}mero de iniciadores reducidos que constituyen el generador. Se puede generalizar la ecuaci\'{o}n anterior para examinar la autosimilitud de varios patrones aleatorios y determinar \(D\).
 
\subsection{Determinacion de la dimensi\'{o}n fractal}
\label{subsec:subseccion2.2}


La determinacion de la dimensi\'{o}n fractal de estructuras en crecimiento, normalmente resulta de la aplicaci\'{o}n directa de las definiciones para D, sin embargo, esto es ineficaz. Es necesario medir o calcular cantidades que se puede demostrar que est\'{a}n relacionadas con la dimensi\'{o}n fractal de los objetos. Existen  tres tipos de enfoques principales para la determinaci\'{o}n de estas cantidades: (i) el experimental, (ii) el te\'{o}rico y (iii) el inform\'{a}tico \cite{Vicsek1992}. El enfoque inform\'{a}tico se obtienen por dos m\'{e}todos principales: 


\begin{enumerate}
\item Digitalizando im\'{a}genes.
\item Procedimientos num\'{e}ricos.
\end{enumerate}
 
  En el segundo caso, los datos generados num\'{e}ricamente suelen producirse mediante variaciones del m\'{e}todo de Monte Carlo. A continuaci\'{o}n, discutiremos c\'{o}mo medir D para un solo objeto. Para hacer las estimaciones m\'{a}s precisas, generalmente se calcula la dimensi\'{o}n fractal para muchos grupos y se promedia sobre los resultados. Quiz\'{a}s, el m\'{e}todo m\'{a}s simple es usar la definici\'{o}n de  D como se indica en \ref{subsec:subseccion2.1}. Por lo tanto, la dimensi\'{o}n fractal se puede obtener ajustando una l\'{i}nea recta a la parte asint\'{o}tica de los datos de  N(R), \textit{v. gr.}, usando el m\'{e}todo de m\'{i}nimos cuadrados \cite{Vicsek1992}.
  
  
\begin{comment}
\subsection{M\'{e}todo de conteo de cajas}

Este m\'{e}todo consiste en dividir el espacio que contiene un patr\'{o}n dado en celdas o p\'{i}xeles de longitud (E), despu\'{e}s se cuenta el n\'{u}mero de celdas o p\'{i}xeles que existen \(N(E)\). El n\'{u}mero \(N(E)\) se mide variando el valor de \(E\) y traz\'{a}ndolo en escala logar\'{i}tmica frente a \(E\), esta es una forma pr\'{a}ctica de comprobar la autosimilitud de un patr\'{o}n determinado. Si los puntos se ajustan a una l\'{i}nea recta, se puede concluir que el patr\'{o}n dado es autosimilar, y la dimensi\'{o}n fractal \(D\) se obtiene del valor de la pendiente de la l\'{i}nea \cite{Mustafa1996}. 
\end{comment}

\subsection{Relaci\'{o}n entre Masa y radio}

La relaci\'{o}n masa-radio es \'{u}til para estimar la dimensi\'{o}n de objetos similares a grupos (redes, vasos sangu\'{i}neos, grupos de agregaci\'{o}n). Consiste en seleccionar un punto de origen en el objeto (normalmente el centro de masa) y contar el n\'{u}mero de part\'{i}culas (masa = p\'{i}xeles) que componen el objeto en un radio del origen. Para un objeto euclidiano bidimensional, la relaci\'{o}n masa-radio es:
\begin{equation}
M(r) \propto r^{2}
\end{equation}

Por ejemplo, el \'{a}rea de un plano bajo discos de tamaño creciente es proporcional al cuadrado del radio del disco de medici\'{o}n. El exponente es por tanto la dimensi\'{o}n (de hecho, un cuadrado es bidimensional), pero la masa de un objeto fractal incrustado en dos dimensiones cambia con un exponente fraccionario:

\begin{equation}
M(r) \propto r^{D} 
\end{equation}

Y la dimensi\'{o}n fractal $D_\textit{{masa-radio}}$ se obtiene de:

\begin{equation}
D_\textit{masa radio} = \frac{log (M(r))}{log (r)}
\end{equation}

Se calcula como la pendiente de la regresi\'{o}n lineal de $log(M(r))$ de $log(r)$ \cite{Mustafa1996}.


\subsection{M\'{e}todo de masa}

Es una medida similar al m\'{e}todo de conteo de cajas, la diferencia principal es el uso de c\'{i}rculos en lugar de celdas o p\'{i}xeles. Consiste en selecionar un punto de origen en el objeto de estudio (que usualmente es el centro de masas) y contar el n\'{u}mero de part\'{i}culas (masa = c\'{i}rculos) que componen 	el objeto en un radio r desde el origen. Para dos dimesiones la relaci\'{o}n entre masa y radio es:

\begin{equation}
\mu(d) = Ad^D
\end{equation}

Donde $\mu(d)$ es el n\'{u}mero de pixeles en una caja de tamaño d, d es el di\'{a}metro longitud de la caja, A es un variable y D es la dimensi\'{o}n fractal\cite{Mustafa1996}. 

\subsection{Radio de giro}

Existen estructuras que crecen mediante la adici\'{o}n sucesiva de part\'{i}culas al objeto. Esta situaci\'{o}n es com\'{u}n, por ejemplo, en simulaciones de Monte Carlo, donde se registra el n\'{u}mero total de part\'{i}culas dentro de un c\'{u}mulo durante el crecimiento. Por lo tanto, es necesario calcular una cantidad llamada radio de giro  \(R_g(N)\) usando la siguiente expresi\'{o}n:

\begin{equation}
R_g = \sqrt{\frac{1}{N} \sum_{i=1}^{N} \| r_i - r_c \|^2}
\end{equation}

donde \(r_i\) es la distancia de la \(i\)\'{e}sima part\'{i}cula desde el centro de masa del c\'{u}mulo y \(N\) es el n\'{u}mero total de part\'{i}culas en el c\'{u}mulo en la etapa dada del proceso de crecimiento. Luego, se asume que:

\begin{equation}
R_g(N) \approx N^{1/D}
\end{equation}

Por lo tanto, \(1/D\) se puede obtener de la pendiente del gr\'{a}fico de \(\ln(R_g)\) como funci\'{o}n de \(\ln(N)\). Esta relaci\'{o}n corresponde a las siguientes suposiciones: i) en el r\'{e}gimen asint\'{o}tico, \(R_g\) es proporcional linealmente al radio total del c\'{u}mulo, ii) las correcciones debido a efectos de borde se pueden ignorar y iii) la estructura no es un multifractal geom\'{e}trico \cite{Mroczka2012, Vicsek1992}.

\subsection{Multifractalidad}

La multifractalidad es una propiedad de ciertos sistemas complejos donde la estructura exhibe autosimilitud a m\'{u}ltiples escalas. En otras palabras, la distribuci\'{o}n de ciertas caracter\'{i}sticas o propiedades del sistema no es uniforme, sino que var\'{i}a de manera heterog\'{e}nea en diferentes escalas espaciales o temporales. Esto significa que diferentes partes del sistema pueden exhibir diferentes grados de autosimilitud fractal, lo que resulta en una multifractalidad.

En el caso espec\'{i}fico del sistemas estudiado por  Carrillo \textit{et al.}\cite{Carrillo2003}, como las dispersiones magnetorreol\'{o}gicas y otros conglomerados formados por procesos de agregaci\'{o}n, la multifractalidad se manifiesta en la variaci\'{o}n de la dimensi\'{o}n fractal a lo largo de diferentes etapas del proceso de formaci\'{o}n de estructuras. Esta variaci\'{o}n indica que la autosimilitud fractal no es constante, sino que cambia a medida que se desarrollan diferentes interacciones y se forman conglomerados de distintas generaciones. Como se observa en el gr\'{a}fico \ref{fig:Carrillo2003} , existen tres porciones claramente distinguibles, asociadas a las tres etapas de agregaci\'{o}n mencionadas anteriormente. Cada porci\'{o}n tiene un comportamiento lineal y ha sido ajustada por una l\'{i}nea recta usando el m\'{e}todo de m\'{i}nimos cuadrados. N\'{o}tese la estrecha similitud entre ambas curvas, mostrando porciones rectas que revelan procesos de agregaci\'{o}n claramente similares que generan el patr\'{o}n jer\'{a}rquico \cite{Carrillo2003}.


\begin{figure}[h]
\vspace{1cm}
\centering
{\includegraphics[width=7cm]{Carrillo2006} \par}
\caption{ $\log_{10}N$ versus $\log_{10}r$ para estructuras formadas en un sistema magnetorreológico, para dos fracciones volumétricas de partículas, la curva inferior $\phi = 0.04$ y la superior $\phi = 0.06$. Los valores de $r$ se expresan en términos del tamaño medio de la partícula $\sigma$. Inserción: $\log_{10}(N_i - N_{i-1})$ versus $\log_{10} r_i$ correspondiente a la curva inferior. Lado derecho, micrografías de la estructura de conglomerado: (a) 700 G y (b) 400 G, ambas con $\phi = 0.06$. (c) 500 G y (d) 400 G, ambas con $\phi = 0.04$. Imagen tomada de Carrillo \textit{et al} \cite{Carrillo2003}.}
\label{fig:Carrillo2003}
\end{figure}

   


\subsection{Geometr\'{i}a fractal en prote\'{i}nas}

Ahora que hemos definido D y explicado algunos m\'{e}todos para obtenerlo, surge naturalmente la pregunta de si el conocimiento de la magnitud de la dimensi\'{o}n tiene alguna utilidad. Las respuestas provienen de varios experimentos en los que se ha empleado para cuantificar algunos aspectos de la morfolog\'{i}a proteica. Dado que la estructura de las prote\'{i}nas tiene una forma tan compleja que solo se puede analizar adecuadamente mediante el enfoque de la geometr\'{i}a fractal \cite{Mustafa1996}. 

Una prote\'{i}na consiste en una cadena de polip\'{e}ptidos compuesta por residuos de amino\'{a}cidos unidos entre s\'{i} por enlaces pept\'{i}dicos. La cadena polipept\'{i}dica o columna vertebral forma un pol\'{i}mero lineal compuesto por unidades repetitivas que son id\'{e}nticas, excepto por los terminales de la cadena \cite{Mustafa1996}.

La estructura primaria de un proteina se conoce como la secuencia de \'{a}cidos am\'{i}nicos y determina la conformaci\'{o}n nativa. Los elementos de la estructura secundaria son las $\alpha$-h\'{e}lices y l\'{a}minas $\beta$, mientras que la estructura terciaria se relaciona con las disposiciones espaciales generales de los residuos de amino\'{a}cidos en las prote\'{i}nas. Por lo tanto, la dimensi\'{o}n fractal se puede utilizar para caracterizar las estructuras terciarias de prote\'{i}nas y enzimas. Tambi\'{e}n se pueden describir mediante diferentes modelos fractales y diferentes tipos de dimensiones fractales. Muchos trabajos pioneros han demostrado una y otra vez que la prote\'{i}na que tiene autosimilitud estad\'{i}stica y puede ser manejada por el enfoque fractal. Aunque se pueden construir fractales iterados que son perfectamente autosimilares, la autosimilitud de una cadena de prote\'{i}nas es cierta en lo com\'{u}n solo en un sentido estad\'{i}stico, es decir, la parte no siempre se ver\'{a} exactamente como el todo. Otra diferencia entre las macromol\'{e}culas biol\'{o}gicas y los objetos ideales es que una fractal no es autosimilar en todas las escalas de longitud. Hay l\'{i}mites de tamaño superior e inferior m\'{a}s all\'{a} de los cuales una macromol\'{e}cula ya no es fractal. Por lo que la investigaci\'{o}n fractal sobre enzimas y prote\'{i}nas es actualmente un campo activo \cite{Mustafa1996}. 



\subsection{Medidas basadas en distancia de la similitud estructural de prote\'{i}nas: RMSD}

Auque el problema de que medir las diferencias entre dos estructuras similares 
parece un problema sencillo, su cuantificaci\'{o}n puede ser compleja 
y sigue evolucionando \cite{Kufareva2012}. Una de las medidas 
cuantitativas m\'{a}s comunes para determinar la similitud entre 
dos conjuntos de coordenadas at\'{o}micas es la Desviaci\'{o}n Cuadr\'{a}tica Media 
(RMSD, por sus siglas en ingl\'{e}s), se expresa como:

\begin{equation}
RMSD = \sqrt{\frac{1}{n} \sum_{i=1}^{n} d_i^2}
\end{equation}

donde \(n\) es el n\'{u}mero total de pares de \'{a}tomos considerados 
y \(d_i\) es la distancia entre los \'{a}tomos del par \(i\).


El uso del RMSD es crucial en campos como la bioqu\'{i}mica 
y la qu\'{i}mica computacional porque permite comparar modelos generados por 
simulaciones o experimentos para verificar su similitud con estructuras 
conocidas, como prote\'{i}nas o complejos moleculares.

No obstante, el RMSD puede ser engañoso cuando se trata de diferencias
 locales significativas. Por ejemplo, dos estructuras que difieren solo 
 en un segmento pequeño, como un bucle o un extremo flexible, pueden tener 
 un RMSD global alto, lo que dificulta el an\'{a}lisis. Por esta raz\'{o}n, encontrar
  m\'{e}todos alternativos o enfoques complementarios para comparar estructuras 
  es fundamental. Estas alternativas pueden enfocarse en las regiones clave,
   permitiendo a los investigadores ignorar las variaciones menores y 
  centrarse en las partes cr\'{i}ticas de las estructuras.

\begin{comment}
Sin embargo, el RMSD tiene una desventaja clave: es muy sensible a
 cambios grandes en segmentos pequeños. Por lo tanto, aunque dos 
 estructuras sean muy parecidas en su mayor\'{i}a, un pequeño cambio, 
 como la rotaci\'{o}n de un bucle o la flexibilidad de un extremo,
  puede resultar en un RMSD alto, complicando la comparaci\'{o}n eficaz.
\end{comment}

\clearpage


\section{Objetivos}
\subsection{Objetivo general}

Determinar si las prote\'{i}nas presentan caracter\'{i}sticas multifractales. De ser as\'{i}, analizar
dicha multifractalidad con el fin de utilizar una o m\'{a}s variantes de la
dimensi\'{o}n fractal como medida(s) caracterizadora(s)
de la estructura y patrones de agregaci\'{o}n de las prote\'{i}nas.


\subsection{Objetivos particulares}

\begin{itemize}

\item Seleccionar un conjunto de prote\'{i}nas de estudio que tengan potencial para aplicaciones posteriores.
Por ejemplo prote\'{i}nas nativas y mutadas, prote\'{i}nas que han mantenido su funci\'{o}n biol\'{o}gica pero que se
han diferenciado estructuralmente, entre otros casos. Se priorizar\'{a}n los casos en que se cuente con informaci\'{o}n
experimental (PDB).

\item Adaptar la determinaci\'{o}n de la dimensi\'{o}n fractal de masa a las prote\'{i}nas, a partir de otros casos como geles o materiales ferromagn\'{e}ticos.

\item Analizar la dimension fractal de masa de un conjunto de prote\'{i}nas, partiendo de la definici\'{o}n de multifractalidad y usando los datos experimentales recabados.

\item Analizar si la multifractalidad de una prote\'{i}na se observa con otras formas de determinar la
dimensi\'{o}n fractal de masa, por ejemplo el radio de giro.

\item Explorar el uso potencial de la dimensi\'{o}n fractal de masa como una herramienta para identificar
o resolver entre prote\'{i}nas gen\'{e}ticamente similares pero con diferencias estructurales notables.

\end{itemize}

\clearpage

\section{Metodolog\'{i}a}

\begin{enumerate}
	
\item Obtenci\'{o}n de la estructura tridimensional de la prote\'{i}na: Obtener la estructura tridimensional de la prote\'{i}nas que fue determinda mediante t\'{e}cnicas experimentales como la cristalograf\'{i}a de rayos X o la espectroscopia de resonancia magn\'{e}tica nuclear.

\item C\'{a}lculo de la dimensi\'{o}n fractal de masa: El c\'{a}lculo de la dimensi\'{o}n fractal de masa implica medir c\'{o}mo var\'{i}a el volumen del objeto (en este caso, la prote\'{i}na) con respecto a la escala a la que se est\'{a} observando. Esto se puede hacer utilizando algoritmos espec\'{i}ficos para el c\'{a}lculo de la dimensi\'{o}n fractal de masa como el algoritmo box-counting.

\item An\'{a}lisis de la complejidad geom\'{e}trica de la prote\'{i}na: Una vez calculada la dimensi\'{o}n fractal de masa para la prote\'{i}na, se analizar\'{a} c\'{o}mo var\'{i}a esta dimensi\'{o}n en diferentes escalas. Esto nos permitir\'{i}a caracterizar la complejidad geom\'{e}trica de la prote\'{i}na a diferentes niveles de detalle.

\end{enumerate}

\clearpage

\section{Cronograma}
\begin{table}[hbp!]
\centering
\footnotesize
\setlength{\tabcolsep}{2.0pt}
\begin{tabular}{||p{0.4\linewidth}|c|c|c|c|c|c|c|c|c|c|c|c||}
\hline
\textbf{Actividad} & \multicolumn{5}{c|}{2023} & \multicolumn{7}{c||}{2024}\\
\hline
& Ago & Sep & Oct & Nov & Dic & Ene & Feb & Mar & Abr & May & Jun & Jul\\
\hline
Materias del primer semestre & X & X & X & X & X & & & & & & &  \\
\hline
LII Winter Meeting on Statistical Physics & & & & & & X & & & & & & \\
\hline
Curso de Qu\'{i}mica Cu\'{a}ntica en \'{a}tomos & & & & & & X & X & X & X & X & X & \\
\hline
Materias del segundo semestre &  &  &  &  &  & X & X & X & X & X & X &\\
\hline
Revisi\'{o}n bibliogr\'{a}fica & X & X & X & X & X & X & X & X & X & X & X & X \\
\hline
Presentaci\'{o}n de protocolo &  &  &  &  &  &  &  &  &  &  & X & \\
\hline
Preparaci\'{o}n de sistemas de estudio  &  &  &  &  &  &  &  &  &  &  &  & X \\
\hline
& \multicolumn{5}{c|}{2024} & \multicolumn{7}{c||}{2025}
\\\hline
& Ago & Sep & Oct & Nov & Dic & Ene & Feb & Mar & Abr & May & Jun & Jul\\
\hline
XXII RMFQT &  &  &  & X &  &  &  &  &  &  &  &\\
\hline
XXlX Simposio interno FCQ - ICUAP &  &  &  & X &  &  &  &  &  &  &  &\\
\hline
LIII Winter Meeting on Statistical Physics &  &  &  &  &  & X &  &  &  &  &  &\\
\hline
Revisi\'{o}n bibliogr\'{a}fica & X & X & X & X & X & X & X & X & X & X & X &\\
\hline
Redacci\'{o}n de tesis &  &  &  &  &  & X & X & X & X &  &  &\\
\hline
Revisi\'{o}n de tesis &  &  &  &  &  &  &  &  & X & X & X & \\
\hline
Examen &  &  &  &  &  &  &  &  &  &  &  & X\\
\hline
%Actividad &  &  &  &  &  &  &  &  &  &  &  &  \\
\hline
\end{tabular}
\end{table}


\clearpage

\bibliographystyle{plain}
\bibliography{bibliografia}

\end{document}

