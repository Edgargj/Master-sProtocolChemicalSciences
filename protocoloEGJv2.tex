%%%%%%%%%%%%%%%%%%%%%%%%%%%%%%%%%%%%%%%%%
% Protocolo Laboratorio de Fisicoquímica Teórica 
% LaTeX Template
% Version 0.1 (10/12/17)
%
% Part of this template was downloaded from:
% http://www.LaTeXTemplates.com
%
% Edgar García Juarez (b613) edgargj.16@gmail.com
%%%%%%%%%%%%%%%%%%%%%%%%%%%%%%%%%%%%%%%%%
%-----------------------------------------------------------------------------% 
% Packages
%-----------------------------------------------------------------------------%
\documentclass[11pt]{article} 	
\usepackage[utf8]{inputenc}
\usepackage[spanish]{babel}
\usepackage{latexsym,amssymb,amsfonts}
\usepackage[usenames]{color}
\usepackage{float}
\usepackage{graphicx}
\usepackage{verbatim}

\usepackage{amsmath,mathtools}
\usepackage[font={small},format=plain,labelfont=bf]{caption} % Size and format of foot of figures and tables
\usepackage{color}
\usepackage{hyperref}
%-----------------------------------------------------------------------------%
% Margin settings 
%-----------------------------------------------------------------------------%
\usepackage{geometry}
\geometry{
	paper=letterpaper, % Paper type
	inner=2.5cm, % Inner margin
	outer=2.5cm, % Outer margin
	top=3.5cm, % Top margin
	bottom=3.5cm, % Bottom margin
}
%-----------------------------------------------------------------------------% 
% New Commands
%-----------------------------------------------------------------------------%
\renewcommand{\tablename}{Tabla} % Name of table foot 
\renewcommand{\baselinestretch}{1.5} % space between lines 1.5
\newcommand\tab[1][0.5cm]{\hspace*{#1}}
\newcommand\vtab[1][0.5cm]{\vspace*{#1}}
%-----------------------------------------------------------------------------% 
%-----------------------------------------------------------------------------%
% Document
%-----------------------------------------------------------------------------%
%-----------------------------------------------------------------------------% 

\begin{document}
%-----------------------------------------------------------------------------%
% Cover Page
%-----------------------------------------------------------------------------% 
\pagestyle{empty} 
\phantom{a}
\vspace{-1.4cm}
\begin{center}
\includegraphics[width=4cm]{buap.pdf}

\Large{\textbf{Benemérita Universidad Autónoma de Puebla}\\
\rule{150mm}{0.1mm}\\
Facultad de Ciencias Químicas\\
Centro de Química - Instituto de Ciencias\\
Posgrado en Ciencias Qímicas}\\
\rule{150mm}{0.1mm}

\vtab[.1cm]
{\scshape\large \textbf{La dimesión fractal  de masa como medida caracterizadora\\de la estructura de las proteínas}}\\
\vtab[.1cm]
\Large{Protocolo de tesis de Maestría}\\
\vtab[0.2cm]

Lic. en Química \\
Édgar García Juárez \\
\vtab[0.5cm]
\begin{tabular}{cc}
\vtab[10mm]
Director de tesis & Co-directora de tesis \\
Dr. J. Manuel Solano Altamirano & Dra. Viridiana Vargas Castro \\ 
FCQ - BUAP & FCQ - BUAP \\ 
\end{tabular} 
\vtab[1cm] \\
{\today}
\end{center}

\clearpage

\section{Introducción}


\section{Antecedentes}


\section{Objetivos}
\subsection{Objetivo general}

Determinar si las proteínas presentan características multifractales. De ser así, analizar y caracterizar
dicha multifractalidad con el fin de utilizar una o más variantes de la
dimensión fractal como medida(s) caracterizadora(s)
de la estructura y patrones de agregación de las proteínas.


\subsection{Objetivos particulares}

\begin{itemize}

\item Seleccionar un conjunto de proteínas de estudio que tengan potencial para aplicaciones posteriores.
Por ejemplo proteínas nativas/mutadas, proteínas que han mantenido su función biológica pero que se
han diferenciado estructuralmente, entre otros casos. Se priorizarán casos en que se cuente con información
experimental (PDB).

\item Adaptar la determinación de la dimensión fractal de masa a las proteínas, a partir de otros casos como geles o materiales ferromagnéticos.

\item Analizar la dimension fractal de masa de un conjunto de proteínas, partiendo de la definición de multifractalidad y usando los datos experimentales recabados.

\item Analizar si la multifractalidad de una proteína se observa con otras formas de determinar la
dimensión fractal de masa, por ejemplo el radio de giro.

\item Explorar el uso potencial de la dimensión fractal de masa como una herramienta para identificar
o resolver entre proteínas genéticamente similares pero con diferencias estructurales notables.

\end{itemize}

\clearpage

\section{Metodología}


\clearpage

\section{Cronograma}
\begin{table}[hbp!]
\centering
\footnotesize
\setlength{\tabcolsep}{2.0pt}
\begin{tabular}{||p{0.4\linewidth}|c|c|c|c|c|c|c|c|c|c|c|c||}
\hline
\textbf{Actividad} & \multicolumn{5}{c|}{2023} & \multicolumn{7}{c||}{2024}\\
\hline
& Ago & Sep & Oct & Nov & Dic & Ene & Feb & Mar & Abr & May & Jun & Jul\\
\hline
Materias del primer semestre & X & X & X & X & X & & & & & & &  \\
\hline
LII Winter Meeting on Statistical Physics & & & & & & X & & & & & & \\
\hline
Curso de Química Cuántica en Átomos & & & & & & X & X & X & X & X & X & \\
\hline
Materias del segundo semestre &  &  &  &  &  & X & X & X & X & X & X &\\
\hline
Revisi\'on bibliogr\'afica & X & X & X & X & X & X & X & X & X & X & X & X \\
\hline
Presentación del protocolo &  &  &  &  &  &  &  &  &  &  & X & \\
\hline
Preparación de sistemas de estudio  &  &  &  &  &  &  &  &  &  &  &  & X \\
\hline
& \multicolumn{5}{c|}{2024} & \multicolumn{7}{c||}{2025}\\\hline
& Ago & Sep & Oct & Nov & Dic & Ene & Feb & Mar & Abr & May & Jun & Jul\\
\hline
Curso de Termodinámica Estadística & X & X & X & X & X &  &  &  &  &  &  &\\
\hline
XXII RMFQT &  &  &  & X &  &  &  &  &  &  &  &\\
\hline
XXlX Simposio interno FCQ - ICUAP &  &  &  & X &  &  &  &  &  &  &  &\\
\hline
LIII Winter Meeting on Statistical Physics &  &  &  &  &  & X &  &  &  &  &  &\\
\hline
Revisi\'on bibliogr\'afica & X & X & X & X & X & X & X & X & X & X & X &\\
\hline
Redacci\'on de tesis &  &  &  &  &  & X & X & X & X &  &  &\\
\hline
Revisi\'on de tesis &  &  &  &  &  &  &  &  & X & X & X & \\
\hline
Examen &  &  &  &  &  &  &  &  &  &  &  & X\\
\hline
%Actividad &  &  &  &  &  &  &  &  &  &  &  &  \\
\hline
\end{tabular}
\end{table}


\clearpage

\bibliographystyle{plain}
\bibliography{bibliografia}

\end{document}

